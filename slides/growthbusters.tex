\documentclass{beamer}
\usepackage[ngerman]{babel}
\usepackage[utf8]{inputenc}
\usepackage{graphicx}
\usepackage{textcomp}
\usepackage{amssymb}

\setbeamertemplate{navigation symbols}{}

\setbeamercolor{normal text}{fg=white}
\setbeamercolor{structure}{fg=white} 
\setbeamercolor{background canvas}{bg=black}

\begin{document}

\title{Warum Wachstum nicht \\ das Maß aller Dinge ist.}
\author{}
\date{}


% Mir ist sehr wohl bewusst, dass ihr hierfür mehr oder weniger die falsche Zielgruppe seid, da ihr eh schon so denkt, sonst wärt ihr heute nicht hier, aber vielleicht schaffe ich es ja dennoch tieferes Interesse zu wecken und womöglich anhand des Videomitschnitts noch mehrere anzusprechen.
\frame{
\vspace{2 cm}
\maketitle
}

% Der Vortrag basiert auf einem wunderbaren Dokumentarfilm eines Typens namens Dave Gardner aus Colorado Springs. Selbsternannter Growthbuster, um in Anlehnung an die Ghostbusters dem Wachstum auf den Pelz zu rücken.
\frame{
\begin{center}
\includegraphics[scale=.4]{dave.png}\\
\vspace{.5cm}
Dave Gardner \\
\emph{Growthbuster}
\end{center}
}

% Was ist das eigentliche Problem? Wir messen uns an Wachstum. Wachstum ist ein großer Bestandteil unseres Lebens, der sehr tief verankert ist.
\frame{
\begin{center}
Wir messen uns an Wachstum.
\end{center}
}

% Doch in Wahrscheit schafft Wachstum weitaus mehr Probleme als es beseitigt.
\frame{
\begin{center}
Wachstum schafft mehr\\ Probleme als es beseitigt.
\end{center}
}

% Fangen wir vorne an.
% Ich zitiere wortwörtlich aus dem 'Wachstumsprogramm' unser aller Lieblingspartei, der FDP.
\frame{
\begin{center}
\emph{Wachstum schafft Freiheit}\\
\vspace{.2cm}
\emph{Wachstum hält uns in Verbindung}\\
\vspace{.2cm}
\emph{Wachstum ermöglicht faire Steuern}\\
\vspace{.2cm}
\emph{Wachstum ist wie Frühling}\\
\end{center}
}
% Aber auch bei anderen sieht es nicht besser aus, Mutti schwärmt auch regelmäßig von Wachstum, die SPD sieht es auch nicht grundlegend anders. Lediglich die Grünen und die Linke scheinen den Begriff überhaupt in Frage stellen zu wollen.

% Aber warum wollen wir immer mehr? 
\frame{
\begin{center}
Warum wollen wir immer mehr? \\
Warum sind wir nie zufrieden?
\end{center}
}

% Es gibt ein altes Sprichwort: If you don't grow, you die.
\frame{
\begin{center}
\emph{If you don't grow, you die.}
\end{center}
}
% In Wahrheit ist Wachstum aber ein ziemlich doofes Konzept um unser Wohlergehen darauf zu basieren. Aber gehen wir mal von einer anderen Seite ran.

% Es gibt etliche Beispiele dafür, dass Nationen ihr Bevölkerunswachstum fördern wollen.
% Australiens Finanzminister hat mal rausgehauen.
\frame{
\begin{center}
\emph{One for mom, one for dad,\\ one for the \textbf{country}}
\end{center}
}
% Es gibt haufenweise Beispiele für finanzielle Anreize um Populationswachstum zu fördern.

% Japan
% Ein Pärchen mit Kindern bekommt umgerechnet 107€/mtl
\frame{
\begin{center}
monatlich 13.000\textyen
\end{center}
}

% Korea
% Das Kindermädchen wird direkt vom Staat bezahlt
\frame{
\begin{center}
vom Staat bezahltes Kindermädchen
\end{center}
}

% Russland
% 7000€ Bonus für jedes Kind nach dem ersten und ein Tag frei um sich 'fortzupflanzen'
\frame{
\begin{center}
7.000€ ab dem dritten Kind \\
und \\
\emph{Fortplanzungsurlaub}
\end{center}
}

% der Inselstaat Palau im Pazifik
% Strafen bis hin zu Gefängnisaufenthalt für die Verwendung von Verhütungsmitteln
\frame{
\begin{center}
Geld oder Gefängnisstrafe für die\\
Verwendung von Verhütungsmitteln
\end{center}
}
% Und warum genau tun wir alles erdenklich möglich um die Entwicklung in diese Richtung zu treiben?

% Uns wird schon immer eingeredet, dass Wachstum gut sei. Seit der Geburt!
\frame{
\begin{center}
\emph{Wachstum ist gut!}
\end{center}
}
% Wachsende Firmen sind erfolgreiche Firmen. Profite sollen wachsen, Einkommenswachstum usw. Wir wollen ja auch, dass unsere Bankkonten und Kapitalanlagen wachsen.
% Die Größe macht einen Unterschied! Wir beurteilen alles nach der Größe, Städte, Firmen, Häuser, Parties. Ist ja wie in der Umkleide in der 9. Klasse. Bzw. passt hier irgendwo offensichtlich auch ein 'that's what she said' witz rein. 
% aber dass wir dieses Konzept auch hinterfragen können, darauf gilt es zu kommen.

% Und das alles nur, weil wir ein simples Konzept nicht verstehen können.
\frame{
\begin{center}
$e^x$
\end{center}
}
% Die Exponentialfunktion

% Bakterienbeispiel
% Ihr seid jetzt Bakterien, nicht abwertend gemeint, es geht mir um die Fortpflanzung
\frame{
\begin{center}
\emph{Stell dir vor du wärst ein Einzeller}
\end{center}
}
% Alleine in einer Flasche
% Jede Minute teilt ihr euch
% das ist stetiges Wachstum
% Wir fangen um 11 an, um 12 ist die Flasche voll
% Jetzt ist die Frage, wann war die Flasche halbvoll?
% Antworten wie 11.30 Uhr erwartet, aber wir haben stetiges Wachstum, es war eine Minute vor 12!
% Du bist ein Bakterium in der Flasche, wann fällt dir auf, dass wir ein Problem haben?
% 5 vor 12 sind gerade mal 3% der Flasche gefüllt, 97% sind freier ungenutzer Platz, der Offenheit und Fortschritt verspricht
% Zurück zur Realität
% Gehen wir mal von 3% Wirtschaftswachstum im Jahr aus, das ist weltweit gesehen ein passender Durchschnittswert
% In 24 Jahren hat sich unser Wirtschaftsfaktor verdoppelt, und nach 24 Jahren nochmal und nochmal
% Nach 'nur' 720 Jahren wären wir bei einer Vermilliardung angekommen!

% Hab hier mal nen netten Graphen mitgebracht. Das ist der Anstieg der Weltbevölkerung in den letzten 10.000 Jahren. Uns, also den Homo Sapiens gibt's schon ne Weile länger, aber das ist egal. 
\frame{
\begin{center}
\includegraphics[scale=.6]{population.png}
\end{center}
}
% Da ist ein Problem auf jeden Fall schon einmal sichtbar...
% Graphen Ölverbrauch, Artensterben usw usw sehen exakt genauso aus!

% Wir überfordern ganz eindeutig die Möglichkeiten unseres Planeten.
\frame{
\begin{center}
Wir überfordern die Möglichkeiten \\
dieses Planeten uns zu tragen.
\end{center}
}

% Es gibt ja das interessante Konzept des ökologischen Fußabdrucks
\frame{
\begin{center}
\emph{Ökologischer Fußabdruck}
\end{center}
}
% jeder Mensch reserviert sich sozusagen einen gewissen Flecken Land auf der Erde, der ausschließlich dafür gedacht ist, Essen nur für dich anzubauen. Und natürlich die Grundlage für jegliche organischen Stoffe bereitzustellen, die du benötigst. Und dann natürlich noch ne Art private Müllhalde, die all das berherbergt, was du wegschmeißt und von der Natur aufgenommen werden muss. Alles zusammen ergibt ein kleines - bzw. gar nicht mal so kleines - Ökosystem, was komplett für deine Erhaltung gedacht ist.
% Momentan leben wir so, als ob wir 1,5 Erden zur Verfügung hätten. Und wie soll das gehen? Wir verbrauchen endliche Ressourcen, das kann man zwar ne Weile machen, aber nicht für immer. Wir klauen diese Ressourcen sozusagen von zukünftigen Generationen.

% Auf den Punkt gebracht ist das Problem also folgendes.
\frame{
\begin{center}
Wachstum: \\
- liquidiert unsere Ressourcen -\\
- ist unfair gegenüber zukünftigen Generationen -\\
- physisch unmöglich aufrecht zu erhalten -
\end{center}
}

% Wie viele von euch glauben, dass es möglich sei endloses Wachstum in einer endlichen Umgebung fortzuführen? Keiner, genau. Und dennoch ist es genau das, was wir glauben tun zu können.
% http://en.wikipedia.org/wiki/William_E._Rees
% \frame{
% \begin{center}
% William E. Rees
% \end{center}
% }

% Was können wir also tun?
\frame{
\begin{center}
Wachstumsorientierte Ziele aufgeben!
\end{center}
}
% Jaja, das ist noch sehr vage, gehen wir mal weiter.

\frame{
\begin{center}
Wachstum $\neq$ Wohlstand
\end{center}
}

% Da gibt's auch noch ein wunderbares Zitat von Albert Einstein.
%\frame{
%\begin{center}
%\emph{Probleme kann man niemals mit derselben\\ Denkweise lösen, durch die sie entstanden sind.}\\
%\vspace{.2cm}
%\hspace{6.5cm} Albert Einstein
%\end{center}
%}

% also, jetzt konkret
\frame{
\begin{center}
Familiengröße verringern
\end{center}
}

% Wichtig ist vor allem, das Problem so direkt anzusprechen.
\frame{
\begin{center}
Das Problem Überbevölkerung ansprechen!
\end{center}
}
% über übervölkerung zu reden bedeutet keine kontrolle, es bedeutet sich gedanken über unsere entscheidungen zu machen, so dass wir diese sinnvoll treffen können um eine bessere zukunft zu schaffen. Es geht nicht darum, Leuten vorzuschreiben wie viele Kinder sie haben sollten.
% wir leben im 21. jahrhundert, wir haben bereits seit einer weile die nötigen werkzeuge um die wahl des kinderlosen daseins zu ermöglichen, aber es gibt immer noch gesellschaftliche erwartungen. Druck durch Eltern, Freunde, Geschwister usw.

% und so kann man letztendlich auch wirklich sagen
%\frame{
%\begin{center}
%Kondome für den Artenschutz!
%\end{center}
%}

% es gibt auch noch einen ansatz, von einem gewissen Peter Victor, Ökologe an der York University, der etliche ausführliche Simulationen zu dem Thema durchgeführt hat. Und zwar kommt er zu dem Schluss, dass selbst mit einer Wachstumsrate von 0% Vollbeschäftigung möglich sei - jaja, das ist ne andere Baustelle, stimme dem auch zu, dass das Konzept nicht mehr zeitgemäß sei, ich gehe hier aber auf die Ansichten von Peter Victor ein -, eine ausschlaggebende Verringerung im Ausstoß von Treibhausgasen möglich sei, Staatsschulden ausgeglichen werden und etliche große Schritte weg von Armutsverhätnissen getätigt werden könnten.
% Und wie? Simpel.
\frame{
\begin{center}
40 Stunden Woche wird zur 21 Stunden Woche.
\end{center}
}
% über die nächsten 10-30 Jahre können wir so auch produktiver werden, von dieser Produktivität aber auch profitieren in dem wir mehr Freizeit haben, die nicht zwangsweise mit dem steigenden Verbrauch von Konsumgütern gefüllt werden muss.
% Wir sind keine gierigen und egoistischen Lebewesen, wie wir von unserer Konsumgesellschaft erzählt bekommen, wir sind weitaus mehr, etwas viel größeres und zu so viel mehr fähig.
% Werdet auch Growthbuster!

% Zum Abschluss gibt's noch diesen tollen kleinen Comic, tolles Teil.
\frame{
\begin{center}
\includegraphics[scale=.5]{comic.png}
\end{center}
}

\end{document}